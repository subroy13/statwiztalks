\documentclass[11pt]{article}

\usepackage[margin=1in]{geometry}
\usepackage{mathaddons}
\usepackage{xcolor}
\usepackage{url,hyperref}
\hypersetup{
    colorlinks=true,
    citecolor=blue,
    linkcolor=blue,
    urlcolor=blue
}
\usepackage{booktabs}

\newcommand{\todo}[1][]{\textcolor{red}{\textbf{TODO:} #1}}


%%%%%%%
% Metadata
\title{Some useful techniques of statistics}
\author{Subhrajyoty Roy}
\date{\today}

%%%%%%%


\begin{document}
\maketitle
\tableofcontents

\newpage
\section{Gaussian Integrals}

Imagine you come across some integral of the form $\int e^{f(x)}$ where $f(x)$ can be a complicated expression, but you see it is quadratic in $x$. You understand that it is a Gaussian integral, but you are too reluctant to complete the square and figure out the mean, variance, etc. so that you can apply the Gaussian integral formula.

One simply way is to take derivatives of $f(x)$ with respect to $x$, and since it becomes linear, it is much easy to deal with.

\begin{align*}
    & f(x) = (x - \mu)^2/2\sigma^2 + r^2\\
    \implies {} & f'(x) = 2(x - \mu)/2\sigma^2\\
    \implies {} & f'(0) = -\mu/\sigma^2\\
    \implies {} & f''(0) = 1/\sigma^2\\
    \implies {} & \mu = -\frac{f'(0)}{f''(0)}, \ \sigma^2 = 1/f''(0)
\end{align*}

Therefore, the Gaussian integral can be written as 
\begin{equation*}
    \int e^{f(x)} = \int \exp\left[ \frac{(x + f'(0)/f''(0))^2}{2 (1/f''(0))} + \left( f(0) - \frac{(f'(0))^2}{2f''(0)} \right) \right] = \exp\left( f(0) - \frac{(f'(0))^2}{2f''(0)} \right)
\end{equation*}

\section{Asymptotic Theory and Taylor's series}

\todo

\section{Conformal Prediction}

\todo 

\section{Fenchel Identity}

\todo





\end{document}